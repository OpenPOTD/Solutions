\SSbreak\\
\emph{Source: Sixth Term Examination Paper I, 2013 Q3}\\
\emph{Proposer: \Pss}\\
\emph{Problem ID: 191}\\
\emph{Date: 2021-05-19}\\
\emph{Difficulty: Easy}\\
\SSbreak
 
\SSpsetQ{Given for any two points $X$ and $Y$, with position vectors
$\textbf x$ and $\textbf y$ respectively, $X*Y$ is defined
to be the point $Z$ along the line $\overleftrightarrow{XY}$ such that $$\dfrac{XZ}{ZY} = \dfrac{\lambda}{1 - \lambda},$$
where $\lambda$ is a fixed number. \medskip

The points $P_1$, $P_2$, $\ldots$
are defined by $P_1 = X*Y$ and, for $n \ge2$, $P_n= P_{n-1}*Y\,.$ Given that $X$ and $Y$ are distinct and that $0<\lambda<1$, the ratio in which $P_n$ divides the line segment $XY$ can be expressed in the form \(1:A\). 
If $\lambda = 0.75$ and $n = 69$, find $\left \lfloor 10^{12} A \right \rfloor$.
  
\begin{center}
\textit{(A scientific calculator may be used.)}
\end{center}
}\bigskip

\begin{solution}\hfil\medskip

 We have
\begin{align*}
    P_1&=X*Y\\
    P_2&=(X*Y)*Y\\
    \cdots&\\
    P_n&=(X*Y)*Y)*Y)...*Y
\end{align*}
Where \(P_n\) has \(n\) \(Y\)'s. Substituting \(X*Y=\lambda\textbf{x}+(1-\lambda)\textbf{y}\) and going through a number of iterations of \(*Y\), we guess \(P_n=\lambda^n+(1-\lambda^n)\textbf{y}\).\\

We now proceed by induction on \(n\). Given we have already verified the case of \(n=1\), let \(n=k,\ k\in\N\) and assume \(P_k=\lambda^k\textbf{x}+(1-\lambda^k)\textbf{y}\). Then for \(n=k+1\), we have \(P_{k+1}=P_k*Y\), and so we have \(P_{k+1}=\lambda^{k+1}\textbf{x}+\lambda(1-\lambda^k)\textbf{y}+(1-\lambda)\textbf{y}\). This simplifies to \(P_{k+1}=\lambda^{k+1}\textbf{x}+(1-\lambda^{k+1})\textbf{y}\), as required. Therefore:

\begin{equation*}
    P_n=\begin{pmatrix}
        \lambda^n \\ 1-\lambda^n
    \end{pmatrix}
\end{equation*}
Now it's simply a case of
\begin{equation*}
    XY:P_n=\begin{pmatrix}
        1\\ 1
    \end{pmatrix} : \begin{pmatrix}
        \lambda^n \\ 1-\lambda^n
    \end{pmatrix}\Rightarrow XY:P_n=1:\frac{\lambda^n}{1-\lambda^n}
\end{equation*}
Hence when \(\lambda=0.75\) and \(n=69\), we have \(\{10^{12}\cdot A\}=\fbox{2394}\)
\end{solution}