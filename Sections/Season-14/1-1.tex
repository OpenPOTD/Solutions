\SSbreak\\
\emph{Source: Original}\\
\emph{Proposer: \Prandom}\\ %\Pchan \Pbrain \Pss
\emph{Problem ID: 189}\\
\emph{Date: 2021-05-17}\\
\emph{Difficulty: Beginner}\\
\SSbreak

\SSpsetQ{
    A 35-by-42 rectangle is rotated around one of its vertices. If $A$ is the area the rectangle sweeps out, find $\lfloor 100A \rfloor.$

    \begin{center}
        \textit{(A scientific calculator may be used.)}
    \end{center}
    %Put Problem Here
}\bigskip

\begin{solution}\hfil\medskip
	
	Noticably after a rectangle is rotated around one of its vertex, it will then form a circle. \\
The radius of the circle is a diagonal because the furthest point from one of its vertex is a diagonal and in a few rotation, it will cover an area of circle with the radius of the diagonal length. \\
Firstly, calculate the radius of the circle by calculating the rectangle diagonal \\
$ r = \sqrt{a^2 + b^2} $ \\
$ r = \sqrt{35^2 + 42^2} $ \\
$ r = \sqrt{2989} $ \\
so the area of the circle of $2989\pi \iff \boxed{939022}$. \medskip

\textit{Remark.} $3.1415926$ and $\frac{355}{113}$ are the least "precise" approximations of $\pi$ needed to get the correct answer.
	%Put sol here
\end{solution}\bigskip
