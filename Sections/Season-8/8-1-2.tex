\SSbreak\\
\emph{Source: \Csmc 2003 Q24}\\
\emph{Proposer: \Pss}\\
\emph{Problem ID: 120}\\
\emph{Date: 2021-02-02}\\
\emph{Difficulty: Beginner}\\
\SSbreak

\SSpsetQ{
Let \(AOB\) be an isosceles right-angled triangle drawn in a quadrant of a circle of radius unit 1. The largest possible circle drawn in the minor segment cut by the line \(AB\) has radius \(r\). The radius of the inscribed circle of the triangle \(AOB\) is \(R\). Given that the value of \(Rr\) can be written in the form \(\frac{a - b\sqrt{c}}{d}\), where \(a,b,c,d\) are positive integers and \(c\) is square-free.\medskip

What is the value of \(a^2 + b^2 + c^2 + d^2\)?
}\bigskip

\begin{solution}\hfil\medskip

\begin{figure}[h!]
    \centering
    \includegraphics{Sections/Files/SMC-2003-Q24}
\end{figure}
Observe that \(\epsilon + 2R + 2r = 1\) and \(\epsilon + 2R = \frac{1}{\sqrt{2}}\), therefore we have \(r = \frac{\sqrt{2}-1}{2\sqrt{2}}\). Then by the sine rule, \(\frac{\sin(45)}{R} = \frac{\sin(90)}{R + \epsilon}\Rightarrow R = \frac{1}{\sqrt{2} + 2}\), this gives \(Rr = \frac{3 - 2\sqrt{2}}{4}\), hence our answer is \(3^2 + 2^2 + 2^2 + 4^2 = \fbox{33}\)
\end{solution}\bigskip
