\SSbreak\\
\emph{Source: Gray Kangaroo 2003 Q24}\\
\emph{Proposer: \Pss}\\
\emph{Problem ID: 126}\\
\emph{Date: 2021-02-08}\\
\emph{Difficulty: Beginner}\\
\SSbreak

\SSpsetQ{
Rui draws 10 points on a large piece of paper, making sure that no three points are in a straight line. He then draws a segment joining each pair of points. If Orlo draws a straight line across Rui's diagram, without going through any of Rui's original points, what is the greatest possilbe number of lines that can be crossed?
}\bigskip

\begin{solution}\hfil\medskip

I claim the answer is \fbox{\(\frac{420^2}{4}=44100\)}.\medskip

For a line drawn across Rui's construction splitting the points so that there are \(n\) on one side and \(420-n\) on the other, clearly there will be \(n(420-n)\) intersections. I.e. \(\frac{420^2}{4}-\left(n-\frac{420}{2}\right)^2\). Thus the number of intersections is maximised when \(n=210\).

\end{solution}\bigskip
