\SSbreak\\
\emph{Source: Original}\\
\emph{Proposer: \Ppi}\\
\emph{Problem ID: 86}\\
\emph{Date: 2021-1-01}\\
\SSbreak

\SSpsetQ{
    In a discord ping fight, $2$ immature friends create a discord server with $5$ channels numbered $1$ through $5$. Then player $1$ and player $2$ both select a (not necessarily distinct) odd numbered channel. Each round the two players ping in their channel.\medskip

    After each round, they move to channel $n+1$ or $n-1$ where $n$ is their current channel with equal probability, and start another round. If the expected number of rounds before both end up on the same channel and make up to each other can be expressed as $\frac{a}{b}$ in lowest terms, find $ab$. \medskip
   
   \emph{Note: If one of the friends is on channel 1 or 5, then the next turn they will be on channel 2 and 4 respectively.}
}\bigskip

\begin{solution}[Write up by \Ppi]\hfil\medskip

    Note that the expected value of rounds it will take if the players start on channel $1$ and $3$ is the same as if they started on $3$ and $5$. Then we just have if the players are on $1$ and $5$ to be a separate case. Let our first state be when players are on $1$ and $5$, the second state be when there are players on $2$ and $4$, the third state be when one is on $1$ and $3$ or $3$ and $5$, and the last state be the end state. We now draw this diagram (forgive me if it looks terrible).

    \begin{figure}[h!]
        \centering
        \begin{tikzpicture}

            \node[state] (1) {State 1}; 
            \node[state, right=of 1] (2) {State 2};
            \node[state, right=of 2] (3) {State 3};
            \node[state, right=of 3, 
                        text=white, 
                        draw=none, 
                        fill=red!50!black] (4) {State 4};
            
            \draw[every loop]
            
                (1) edge[bend left, auto=left]  node {1} (2)
                (2) edge[bend left, auto=left]  node {$\frac{1}{4}$} (4)
                (2) edge[bend right, auto=left]  node {$\frac{1}{2}$} (3)
                (3) edge[bend right, auto=right]  node {$\frac{1}{2}$} (2)
                (3) edge[bend right, auto=left]  node {$\frac{1}{2}$} (4)
                (2) edge[bend left, auto=right] node {$\frac{1}{4}$} (1);
                    
            \end{tikzpicture}
    \end{figure}
    
    So we basically just solve for the expected value from each state and go from there. 
    
    
    Setting up a system of equations where $S_n$ denotes the expected value to reach state $4$ from $n$, we have
    
    $$S_1 = 1 + S_2$$
    $$S_2 = 1+\frac{1}{4}S_1 + \frac{1}{2}S_3 + \frac{1}{4} S_4$$
    $$S_3 = 1+\frac{1}{2}S_2 + \frac{1}{2}S_4$$
    $$S_4 = 0$$
    
    So we have $S_1 = \frac{9}{2}, S_2 = \frac{7}{2}, S_3 = \frac{11}{4}$. 
    Now we do some simple casework.
    
    Case 1: It starts on state $1$.
    This happens with probability $\frac{2}{9}$ so the expected number of rounds minus the one at the end from state $1$ is $1$.
    
    Case 2: It starts on state $3$. 
    This happens with probability $\frac{4}{9}$ so the expected number of rounds minus the one at the end from state $3$ is $\frac{11}{9}$.
    
    So by linearity of expectation, the expected value of rounds before the end is $\frac{20}{9}$. So $ab=\boxed{180}$

\end{solution}\bigskip
