\SSbreak\\
\emph{Source: Senior Mathematics Challenge, 2016 Q17}\\
\emph{Proposer: \Pss}\\
\emph{Problem ID: 75}\\
\emph{Date: 2020-12-21}\\
\SSbreak

\SSpsetQ{
\(A02\) has to choose a three-digit code for his bike lock. The digits can be chosen from 1 to 9. To help him remember them,  \(A02\) chooses three different digits in strictly increasing order, for example \(123\). How many such codes can be chosen?
}\bigskip

\begin{solution}\hfil\medskip
  
If we take 3 numbers from \(\{1,2,3,\cdots,9\}\) there is exactly one possible valid combination. Thus, we have a bijection between valid codes and choosing 3 digits from 9. So the answer is \(\binom{9}{3}=\fbox{84}\).

\end{solution}\bigskip
