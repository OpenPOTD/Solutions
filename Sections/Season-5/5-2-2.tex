\SSbreak\\
\emph{Source: Folklore}\\
\emph{Proposer: \Pss}\\
\emph{Problem ID: 83}\\
\emph{Date: 2020-12-29}\\
\SSbreak

\SSpsetQ{
  If we write the numbers 99999 down to 1 in the following string:

  \begin{center}
    \(999999999899997\ldots10987654321\)
  \end{center}

 What is the \(42069^{th}\) digit multiplied by the \(42070^{th}\) digit?\\

 \begin{center}
   \emph{Note: In the given string, we consider the first 9 on the left to be the first digit and 1 to be the last digit.}
 \end{center}
}\bigskip

\begin{solution}\hfil\medskip

Observe that each number in the string can be split up into their respective numbers by separating them by 5 for all digits greater than 9999, 4 for all digits greater than 999, and less than 100000, and so on. 

\begin{equation*}
  \underbrace{99999}_{5}\underbrace{99998}_{5}\underbrace{99997}_{5}\ldots\underbrace{10}_2\cdots
\end{equation*}

Since there are \((99999-9999)\cdot5=450000\) digits from five-digit numbers contained within the string, we know that the \(42069^{th}\) digit will be within a five-digit number. Since 42069 is one less than a multiple of 5, we can deduce that it will be the 4th digit in a five-digit number. \\
We note that the \(5n-4\)th to \(5n\)th digit belong to the number \(100000-n\) (by either Engineer's or mathematical induction). \\
Thus these two digits are the fourth and fifth digits of \(91586\) respectively, meaning their product is \(8 \cdot 6 = \boxed{48}\). 
\end{solution}\bigskip
