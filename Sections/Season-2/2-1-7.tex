\SSbreak \\
\emph{\Ccmo, 2005 Day 2 P6}\\
\emph{Proposer: \Pss}\\
\emph{Problem ID: 55}\\
\emph{Date: 2020-11-23}\\
\SSbreak
   
\SSpsetQ{ 
    Define functions \(f\) and \(g\) such that \(f(a,b)=2^a3^b\), and \(g(c,d)=5^c7^d\), for \(a,b,c,d\in\Z_{\geq0}\).

    Given \(f(a,b)=1+g(c,d)\), what is the sum of all valid \(b\)'s, \(c\)'s and \(d\)'s, multiplied by the sum of all valid \(a\)'s?
    
    For example if we had valid solutions of \((a,b,c,d)=(1,1,2,4),(5,1,6,2),(0,0,0,0)\)

    Then the answer would be \((\underbrace{1+1+0}_{b's}+\underbrace{2+6+0}_{c's}+\underbrace{4+2+0}_{d's})\times(\underbrace{1+5+0}_{a's})=96\)
}\bigskip
   
\begin{solution}\hfil\medskip 

    We proceed by considering parity, for \(2^a3^b=5^c7^d+1\), we have the RHS as even, thus we must have \(a\geq 1\). 
    If we let \(b=0\), then for \(2^a-5^c\cdot7^d=1\), we have \(2^a\equiv 1\mod{5}\) for \(c\ne0\). 
    This gives \(a\equiv0\mod{4}\), so \(2^a-1\equiv0\mod{3}\). 
    But this clearly cannot be the case so we must have \(c=0\) when \(b=0\). 

    Hence, we consider \(2^a-7^d=1\). 
    Bashing gives \((a,d)=(1,0),\ (3,1)\). 
     hese are the only such solutions as for \(a>4,\ 7^d\equiv -1\mod{16}\), but this is impossible. 
    So for the case of \(b=0\) all possible non-negative integer solutions are \((1,0,0,0),\ (3,0,0,1)\).\\

    Now let \(b>0\) and \(a=1\), so we now consider \(2\cdot3^b-5^c\cdot7^d=1\) under\(\mod{3}\), which gives \(-5^c7^d\mod{3}\). 
    Since \(7^d\equiv 1\mod{3}\), for all \(d\geq0\), we are left with \((-1)^c5^c\equiv 1\mod{3}\).
    Now \(5^c=\{1,2\}\mod{3}\), thus we see that we must have \(c\) being odd. Under\(\mod{5}\), we see that \(2\cdot3^b\equiv 1\mod{5}\), \(3^{b-1}\equiv 1\mod 5\). 
    As we observe that \(3^b\equiv\{3,4,2,1\}\mod{5}\), we must have \(b\equiv 1\mod{4}\). 
    If \(d\ne0\), then \(2\cdot3^b\equiv 1\mod{7}\). Again observe that \(3^b\equiv\{3,2,6,4,5,1\}\mod{7}\), we see \(b\equiv 4\mod{6}\). 
    But \(b\equiv 1\mod{4}\), so a contradiction arises, and thus \(d=0\) and hence \(2\cdot3^b-5^c=1\). 
    For \(b=1\), clearly \(c=1\). 
    So if \(b\geq 2\), then \(5^c\equiv-1\mod{9}\Rightarrow c\equiv 3\mod{6}\). 
    Therefore \(5^c+1\equiv0\mod{(5^3+1)}\Rightarrow 5^c+1\equiv0\mod{7}\), but this contradicts the fact that \(5^c+1=2\cdot3^b\). 
    Hence in this case we only have one solution \((a,b,c,d)=(1,1,1,0)\).\\

    Finally, consider the case where \(b>0\), and \(a\geq 0\). 
    Then we have \(5^c7^d\equiv-1\mod{4}\), and \(5^c7^d\equiv-1\mod{3}\), i.e. \((-1)^d\equiv -1\mod{4}\) and \((-1)^c\equiv -1 \mod{3}\). 
    Therefore we have that both \(c\) and \(d\) being odd. Thus, \(2^a3^b=5^c7^d+1\equiv4\mod 8\). 
    So \(a=2\) and thus \(4\cdot 3^b\equiv 1\mod{5}\) and \(4\cdot3^b\equiv 1\mod{7}\). 
    This gives \(b\equiv2\mod{12}\). 
    Substituting \(b=12k+2\) for \(k\in\Z_{\geq0}\), then \(5^c7^d=(2\cdot3^{6k+1}-1)(2\cdot3^{6k+1}+1)\).

    Now as \(\mathrm{gcd}(2\cdot3^{6k+1}+1,2\cdot3^{6k+1}-1)\), \(2\cdot3^{6k+1}-1\equiv0\mod{5}\), therefore \(2\cdot3^{6k+1}-1=5^a\) and \(2\cdot3^{6k+1}=7^d\). 
    If \(k\geq1\), then \(5^c\equiv-1\mod{9}\). 
    But this is impossible, so if \(k=0\), then \(b=2,\ c=1\), and \(d=1\).  
    Thus in this case, we have only one solution: \((a,b,c,d)=(2,2,1,1)\).

    Hence we can conclude all non-negative integer solutions are

    \begin{equation*}
        (a,b,c,d)=
        \begin{cases}
            &(1,0,0,0)\\
            &(3,0,0,1)\\
            &(1,1,1,0)\\
            &(2,2,1,1)
        \end{cases}
     \end{equation*}
    
    This then gives us an answer of \(\underbrace{(0+0+0+0+0+1+1+1+0+2+1+1)}_{7}\times\underbrace{(1+3+1+2)}_{7}=\fbox{49}\) 
 
\end{solution}\bigskip 