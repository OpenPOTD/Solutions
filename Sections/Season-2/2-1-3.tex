\SSbreak\\
\emph{Source: \Cmat, 2012 Q5}\\
\emph{Proposer: \Pss}\\
\emph{Problem ID: 48}\\
\emph{Date: 2020-11-18}\\
\SSbreak

\SSpsetQ{
    In his evil mechatronics laboratory, Brainy has built a physical manifestation of MODSbot.
	MODSbot's movement is defined by three inputs: \textbf{F} to move forward a unit distance, \textbf{L} to turn left \(90^{\circ}\), and \textbf{R} to turn right \(90^{\circ}\).

    We define a program to be a sequence of commands.
	The program \(P_{n+1}\) (for \(n\geq 0\)) involves performing \(P_n\), turning left, performing \(P_n\) again, then turning right:\medskip

    \[P_{n+1}=P_n\textbf{L}P_n\textbf{R},\ P_0=\textbf{F}\]\medskip

    Unbeknownst to Brainy, MODSbot, though limited in movement, is sentient and realises Brainy is just a small asian Frankenstein, whose intentions for them were nefarious and non-consensual.
	As a result, after Brainy goes home for the day, MODSbot makes its escape from Brainy's laboratory.\medskip

    Let \((x_n,y_n)\) be the position of the robot after performing the program \(P_n\), so \((x_0,y_0)=(1,0)\) and \((x_1,y_1)=(1,1)\), etc.\medskip
   
    How far away from the place Brainy left it does MODSbot make it after performing\(P_{24}\)?
}\bigskip


\begin{solution}\hfil\medskip
            
    Note first that after each iteration of $P_n$ MODSbot faces in the positive $x$ direction, as each $P_n$ contains as many \textbf{L}s as it does \textbf{R}s.
	Now, assuming MODSbot is at $(x_n,y_n)$ after having performed $P_n$, we see the next iteration of $P$ puts MODSbot at $(x_n-y_n,x_n+y_n)$.
	Note then that:
            
        \begin{align*}
            (x_{n+2},y_{n+2}) &= (x_{n+1}-y_{n+1},x_{n+1}+y_{n+1}) = (-2y_n,2x_n)\\
            (x_{n+4},y_{n+4}) &= (-2y_{n+2},2x_{n+2})  = (-4x_n,-4y_n)\\
            (x_{n+8},y_{n+8}) &= (-4x_{n+4},4y_{n+4})  = (16x_n,16y_n)
        \end{align*}
        
    Thus, we see that $(x_{8k},y_{8k}) = (16^k,0)$, and therefore that $\abs{P_{24}} = \fbox{4096}$
\end{solution}\bigskip

\begin{solution}[Write up by \Paiya]\hfil\medskip
        
    Observe that each program has the same amount of left and right turns, so MODSbot will always be facing the positive \(x\)-direction after each program.
	This means that \(\textbf{L}P_n\) is just the program \(P_n\) performed at a 90-degree counterclock-wise rotation.
	For instance \(P_1\) moves MODSbot right 1 and up 1, so \(\textbf{L}P_1\) moves MODSbot up 1 and left 1 (right gets rotated 90 counterclockwise to up and up to left).
	This motivates us to work in the complex plane; let \(P_n\) be the complex-number representing MODSBOT's displacement after following \(P_n\).
	Then \(\textbf{L}P_n=iP_n\), so \(P_{n+1}=P_n+\textbf{L}P_n=(1+i)P_n=\sqrt{2}e^{\frac{\pi i}{4}}P_n\).
	WIth \(P_0=1\) we get \(P_n=2^{\frac{n}{2}}e^\frac{\pi i n}{4}\). 
	So \(|P_{24}=\fbox{4096}\)
\end{solution}\bigskip