\SSbreak\\
\emph{Source: \Cop}\\
\emph{Proposer: \Pkee}\\
\emph{Problem ID: 59}\\
\emph{Date: 2020-11-26}\\
\SSbreak

\begin{mdframed}[backgroundcolor=pagegray,rightline=false,leftline=false, topline=false, bottomline=false]
    \color{white}
    Let \textcolor{qpurple}{$n$} be a positive integer.
    \par
    The \textcolor{qlorange}{p-value} of \textcolor{qpurple}{$n$}, denoted $\persistk{n}$: \\

    The number of digit-sums needed to reduce \textcolor{qpurple}{$n$} to a single digit.
    \par
    Examples: \\

    $\textcolor{qpurple}{69} \rightarrow \textcolor{qpurple}{6} + \textcolor{qpurple}{9} \rightarrow \textcolor{qblue}{15} \rightarrow \textcolor{qblue}{1} + \textcolor{qblue}{5} \rightarrow \textcolor{qgreen}{6}$ needs two digit-sums, so $\persistk{69} = 2$. \\
    $\textcolor{qpurple}{203} \rightarrow \textcolor{qpurple}{2} + \textcolor{qpurple}{0} +  \textcolor{qpurple}{3} \rightarrow \textcolor{qblue}{5}$ needs only a single digit-sum, so $\persistk{203} = 1$. \\
    Clearly $\persistk{5} = 0$.\\

    \par
    Let $\persistsetk{k}$ be the set of all \textcolor{qpurple}{$n$} such that $\persistk{n} = \textcolor{qlblue}{k}$. Given that $\textcolor{qred}{a}, \textcolor{qorange}{b}, \textcolor{qyellow}{c} \in \mathbb{N}$, and
    \begin{equation*}
        \min{(\persistsetk{5})} = \textcolor{qred}{a}\times 10^{\textcolor{qorange}{b}} - \textcolor{qyellow}{c}
    \end{equation*}
    What is the value of $\min(\textcolor{qred}{a} + \textcolor{qorange}{b} + \textcolor{qyellow}{c}) \pmod{\min{(\persistsetk{3})}}$ ?
\end{mdframed}
\color{black}
\bigskip

\begin{solution}
    
\end{solution}