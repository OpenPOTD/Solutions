\SSbreak\\
\emph{Source: \Cbmoo, 2010 P1}\\
\emph{Proposer: \Pss}\\
\emph{Problem ID: 40}\\
\emph{Date: 2020-11-17}\\
\SSbreak
        
\SSpsetQ{
    Brainy has a set of integers, from 1 to \(n\), which he likes to play with.
	Tony Wang, upon seeing the happiness that this set of integers brings Brainy, decides to steal one of the numbers in it.
	Suppose the average number of the remaining elements in the set is \(\frac{163}{4}\).
	What is the sum of the elements in Brainy's set multiplied by the element that Tony stole?
                
    \begin{center}
        \emph{(A four-function calculator may be used)}
    \end{center}
}\bigskip
   
\begin{solution}\hfil\medskip 
   
    We can set up the problem statement as 
    
    \begin{equation*}
        \frac{\frac{n}{2}(n+1)-x}{n-1}=\frac{163}{4}
    \end{equation*}
                
    Where \(x\) is the number Tony has stolen.
	This simplifies to \(4x=2n^2-161n+163\).
	Since \(x\) must be a number within the set \(\{1,2,\ldots,n-1,n\}\), we have that \(1\leq x\leq n\Rightarrow 4\leq 2n^2-161n+163\leq4n\).
	By considering the lower bound, we get $(2n-159)(n-1)\geq 0$.
	This means that \(n\leq 1\Rightarrow n=1\), or \(n\geq \frac{159}{2}\Rightarrow n\geq 80\).
	By similar methodology when considering the upper bound, we get \(1\leq n\leq 81\).
	Thus \(n\in\{1,80,81\}\).
	Clearly, \(n\ne 1\), so either \(n=80\) or \(n=81\).
	Notice that if \(n\) is even, then for \(4x=2n^2-161n+163\) the parity of he RHS is Odd, while the LHS is even, thus a contradiction occurs.
	This means that \(n=81\) and so \(x=61\). Thus the answer is \(\frac{81(82)}{2}\cdot61=\fbox{202581}\).
\end{solution}\bigskip