Welcome to the OpenPOTD solutions booklet! Here you'll find answers \& solutions to all past seasons.\medskip

Solutions for season one were entirely written up by {\fontfamily{qcr}\selectfont Brainysmurfs\#2860}, while {\fontfamily{qcr}\selectfont .19\#9839} has overseen most of season two. 
From season three onwards, Yuchan ({\fontfamily{qcr}\selectfont Angry Any\#4319}) has also been contributing problem proposals and solutions. 

Where possible, from season two onwards, We have tried to include the officially provided solutions to problems, 
or adapted them in line with any changes to the problem statement, and in most cases also and filled in the gaps as best I could, 
to make solutions more approachable to beginners. 
For the more well known questions that are featured in a season (namely problems from the International Mathematical Olympiad), 
instead of providing our own solution, we have included the \emph{Art of Problem Solving} forum post on the question, 
which will contain multiple solution write-ups, as well as discussion about the problem.

In many circumstances problems have not come with official write-ups - or indeed write-ups of any kind - 
and thus have required us to provide our own. In these cases we humbly apologise for any mistakes (or fakesolves!) in advance. 
If you do notice any mistakes, check out How to contribute.\medskip

\addcontentsline{toc}{subsection}{How to Contribute}
\subsection*{How to Contribute}
\label{sec:contribute}

If you would like to contribute to the project - be that through correcting a mistake in the document, 
wanting to contribute a solution write up - or even proposing a problem for a future season, here's how.\medskip

\addcontentsline{toc}{subsubsection}{Correcting Mistakes}
\subsubsection*{Correcting Mistakes}
\label{sec:mistakes}

If you notice any mistakes while going through the solutions document, be it a typo or missing or incorrect information about a particular problem, feel free to submit a push request with the fix. 
If you don't feel comfortable doing that, you can always contact us (see the \hyperref[sec:contact]{Contact Us} to find out how), 
and we'll be happy to fix the error. Alternatively, you can open up an \href{https://github.com/OpenPOTD/Solutions/issues}{Issue} on the GitHub, or mention it in the \href{https://github.com/OpenPOTD/Solutions/discussions}{discussions tab}.\medskip

\addcontentsline{toc}{subsubsection}{Contributing Solutions}
\subsubsection*{Contributing Solutions}
\label{sec:solutions}

Similarly to correcting any mistakes, if you would like to contribute a write-up to a particular problem, 
you can submit a push containing the solution - doing as Romans do (i.e. just look at how others have submitted write-ups and copy that). 
If you are submitting a push, make sure you edit {\fontfamily{qcr}\selectfont preamble.sty} to include your Discord information in a macro 
(scroll to the bottom of the file and you'll see), so that you can include yourself in the contributors list. 
Furthermore, make sure that you credit your solution with {\fontfamily{qcr}\selectfont[Write up by ...]}. 
If any of that sounds complicated or you forget to add that information, that's fine - we'll add it for you.

If creating push requests and fiddling with LaTeX isn't your thing, we'll gladly help you type it up
 if you write it out in plaintext or whatever medium you feel most comfortable using (so long as we can understand it!) - 
 to do this just send it to us using any of the options in \hyperref[sec:contact]{Contact Us}. 
 Similarly, you can use the GitHub \href{https://github.com/OpenPOTD/Solutions/discussions}{discussions tab} and post it there.\medskip 

\textbf{\emph{Note: feel free to submit alternative solutions to any past problems, be it from season 1, or the most recent}}\medskip

\addcontentsline{toc}{subsubsection}{Problem Proposals}
\subsubsection*{Problem Proposals}
\label{sec:problems}

If you'd like to submit a problem proposal - be it an original problem, or just a particular problem you found interesting - 
we're always on the lookout for new problems! For original problems, please ensure you submit the problem with a solution. 
As with solution write-ups, though sending us a {\fontfamily{qcr}\selectfont .tex} file is preferred, it's completely fine to just send a plaintext write up, or a screenshot etc. 
and we can deal with it from there. The same goes for non-original problems, though solutions aren't required, 
they would be greatly appreciated. If you are submitting a non-original problem please ensure you include the problem source. 

Please do not use a public medium to submit a problem proposal - messaging one of us on Discord would
 be the preferred method of communication (See \hyperref[sec:contact]{Contact Us}).\medskip

\addcontentsline{toc}{subsubsection}{Feature Suggestions}
\subsubsection*{Feature Suggestions} 
\label{sec:suggestions}

If you have any ideas when it comes to improving the bot or project, the best place to do that is in the {\fontfamily{qcr}\selectfont \#Suggestions} channel, or any of the other methods listed in \hyperref[sec:contact]{Contact Us}, such as using the GitHub \href{https://github.com/OpenPOTD/Solutions/discussions}{discussions tab}.\medskip

\addcontentsline{toc}{subsection}{Contact Us}
\subsection*{Contact Us}
\label{sec:contact} 

The best way to contact us is through the \href{https://discord.gg/GsPSSHdhPB}{OpenPOTD Discord server}, 
however, you may also contact us through the \href{https://github.com/OpenPOTD/Solutions/discussions}{discussions tab} on Github.
 Alternatively, you can DM any of us on Discord:

 \begin{itemize}
     \item sjbs\#9839 (434767660182405131)
     \item brainysmurfs\#2860 (281300961312374785)
     \item Angry Any\#4319 (580933385090891797)
 \end{itemize}
 \medskip

\addcontentsline{toc}{subsection}{Contributors}
\subsection*{Contributors}
\label{sec:contributors}

Thank you (in no particular order) to the following contributors:
\footnote{The numbers are in the format: Season.Week.Problem}\medskip

\Paiya: Solution write-ups (\hyperref[2-1-1]{2.1.1}, \hyperref[2-1-3]{2.1.3}, \hyperref[2-1-4]{2.1.4}, \hyperref[5-1-4]{5.1.4}, \hyperref[6-1-2]{6.1.2}), Original problem proposal (\hyperref[6-1-3]{6.1.3}) \\
\Ptony: Original problem proposal (\hyperref[1-1-2]{1.1.2})\\
\Ppi: Original problem proposal (\hyperref[1-1-6]{1.1.6}, \hyperref[4-1-1]{4.1.1},\hyperref[4-1-2]{4.1.2},\hyperref[4-1-3]{4.1.3},\hyperref[4-1-4]{4.1.4},\hyperref[4-1-5]{4.1.5}, \hyperref[5-2-5]{5.2.5})\\
\Pbfan: Original problem proposal (\hyperref[1-1-7]{1.1.7})\\
\Pkiesh: Original problem proposal (\hyperref[1-2-3]{1.2.3})\\
\Pchris: Problem proposal (\hyperref[2-1-6]{2.1.6})\\
\Pkee: Original problem proposal (\hyperref[2-2-3]{2.2.3})\\
\PSlas: Solution write-up (\hyperref[2-2-5]{2.2.5})\\
\Parjun: Typo fixes\\
\Pnjoy: Problem proposal (\hyperref[5-2-2]{5.2.2}, \hyperref[5-2-4]{5.2.4}, \hyperref[7-1-3]{7.1.3}), Original problem (\hyperref[5-2-3]{5.2.3}, \hyperref[7-1-2]{7.1.2})\\
\Paops: Typo fixes\\