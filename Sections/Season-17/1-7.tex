\SSbreak\\
\emph{Source: Original}\\
\emph{Proposer: \Pmatt}\\ %\Pchan \Pbrain \Pss
\emph{Problem ID: 238}\\
\emph{Date: 2021-07-25}\\
\emph{Difficulty: Challenging}\\
\SSbreak

\SSpsetQ{

The new MODS logo is an infinitely extended M; that is, a non-self-intersecting polyline made of a ray, two segments, and a ray, in that order (the two outside 
"legs" of the M extend infinitely but the two segments that make up the middle "valley" don't). What is the maximum number of regions the plane can be partitioned into by 
200 MODS logos?
}\bigskip

\begin{solution}\hfil\medskip
  
    The formula in general is $8N^2 - 7N + 1$. Answer: \fbox{318601} \medskip

    We would like to convert this problem into graph form, to use Euler's characteristic formula on. 
    To get rid of the infinite condition, we draw a circle with a sufficiently large enough radius 
    such that the circle intersects only the infinitely long lines, "cutting off" the infinitely 
    long lines there and letting those intersection points with the circle, as well as all other 
    intersection points, be vertices. (For example, for $N = 1$ we have $V = E = 5$.) \medskip

    Our aim is to minimize $V - E$, or, maximize $E - V$. Suppose there are already $N$ MODS 
    logos; we now add one more. For each new intersection added, we turn two intersecting segments 
    into four; since we added a vertex and two edges $E - V$ goes up by 1. To maximize the number of 
    intersections, we let each of the four segments of the new logo intersect all $4N$ original 
    segments (this is made rigorous by something something lines are continuous) for a total change of 
    $16N$. Now, the new MODS logo itself adds four edges (each line in the M) and five vertices 
    (three for the M's arches and two for the intersection for the big circle) for a total 
    difference of $16N - 1$. When $N = 1$ we have a maximum of two regions; induction finishes it off. 
\end{solution}\bigskip
