\SSbreak\\
\emph{Source: Original}\\
\emph{Proposer: \Ptan}\\
\emph{Problem ID: 244}\\
\emph{Date: 2021-07-31}\\
\emph{Difficulty: Challenging}\\
\SSbreak

\SSpsetQ{

	2021 distinct rooms are arranged in a circle, each with 4 doorways, 2 leading to neighboring rooms, 
	1 leading inside the circle, 1 leading outside. Each doorway can only be passed through in one direction. 
	Find the last 10 digits of the number of ways to arrange their directions so that one can access any area 
	(a room, inside the circle, or outside) from any other area.
}\bigskip

\begin{solution}[Write up by \Pqs]\hfil\medskip

    The regions and doorways can be modelled as faces and directed edges of a prism, 
	where one can go from region $A$ to $B$ if the corresponding edge points clockwise w.r. to face $A$. \\ \\
	Suppose there is a directed cycle, then one can only cross from one side of it to the other, 
	thus not fulfilling the criteria of being able to reach any region from any other region. On the contrary, 
	if there are no cycles, then for any nonempty strict subset $S$ of faces, one can always find a face $\notin S$ 
	reachable from some face $\in S$. By selecting a set $\{F\}$ of any face and continuously adding reachable faces to it, 
	one can conclude that any face can be reached from any other face. Thus the answer is the number of ways to 
	direct the edges of a 2021-sided prism to form an acyclic graph. \\ \\
	Let $f(G)$ be the number of ways to direct the edges of an undirected graph $G$ to form an acyclic graph. 
	Let $G\backslash e$ and $G/e$ denote the removal and contraction (removal of edge and merging of 2 vertices) of edge $e$ 
	respectively. Note that $f(G) = 0$ if $G$ contains a loop (edge connecting vertex to itself), 
	and the removal of an edge doesn't change the value of $f$ if there is another edge connecting the same 2 vertices. 
	For some directed graph $G$ with non-loop edge $e$, suppose reversing the direction of $e$ changes whether $G$ is acyclic, 
	then $G\backslash e$ is acyclic and $G/e$ is cyclic. Suppose $G$ is acyclic/cyclic regardless of the orientation of $e$, 
	then both $G/e$ and $G\backslash e$ are acyclic/cyclic. Thus we have $f(G) = f(G/e) + f(G\backslash e)$. \\ \\
	Let $Q_s$ denote an undirected graph in the shape of an $s-1$ by 1 grid of squares,
	containing vertices $(x, y)$ for all pairs of integers $1 \leq x \leq s$, $0 \leq y \leq 1$.
	Let $G+e$ denote the addition of edge $e$ and $G[A,B]$ the merging of vertices $A,B$. Observe $f(Q_1) = 2$ and 
	\begin{align*}
	f(Q_{n+1}) &= f\left(Q_{n+1}\backslash\overline{(n+1,0)(n+1,1)}\right) + f\left(Q_{n+1}/\overline{(n+1,0)(n+1,1)}\right)\\
	&= 4f(Q_n) + (2f(Q_n) + f(Q_n))\\
	&= 7f(Q_n)
	\end{align*}
	Therefore, $f(Q_n) = 2\cdot 7^{n-1}$. Notice $f(Q_2[(1,0),(n,0)]) = 0$, $f(Q_2[(1,0),(n,1)]) = 4$, and
	\begin{align*}
	&\ f(Q_{n+1}[(1,0),(n+1,0)])\\
	=&\ f\left(Q_n+\overline{(1,0)(n,0)}+\overline{(n,1)(n+1,1)}\right) + f\left(Q_n+\overline{(1,0)(n,0)}+\overline{(1,0)(n,1)}\right)\\
	=&\ 3f\left(Q_n+\overline{(1,0)(n,0)}\right) + f(Q_n[(1,0),(n,1)])\\
	=&\ 6\cdot 7^{n-1} + 3f(Q_n[(1,0),(n,0)]) + f(Q_n[(1,0),(n,1)])
	\end{align*}
	\begin{align*}
	&\ f(Q_{n+1}[(1,0),(n+1,1)])\\
	=&\ f\left(Q_n+\overline{(1,0)(n,1)}+\overline{(n,0)(n+1,0)}\right) + f\left(Q_n+\overline{(1,0)(n,0)}+\overline{(1,0)(n,1)}\right)\\
	=&\ 3f\left(Q_n+\overline{(1,0)(n,1)}\right) + f(Q_n[(1,0),(n,0)])\\
	=&\ 6\cdot 7^{n-1} + 3f(Q_n[(1,0),(n,1)]) + f(Q_n[(1,0),(n,0)])
	\end{align*}
	Thus, $f(Q_n[(1,0),(n,0)]) = \frac{1}{28} (8\cdot 7^n - 7\cdot 2^n (3\cdot 2^n + 2))$.
	Finally, $f\left(Q_2+\overline{(1,0)(2,0)}+\overline{(1,1)(2,1)}\right) = 14$ and
	\begin{align*}
	&\ f\left(Q_n+\overline{(1,0)(n,0)}+\overline{(1,1)(n,1)}\right)\\
	=&\ f\left(Q_n+\overline{(1,0)(n,0)}\right) + f\left(Q_n+\overline{(1,0)(n,0)}[(1,1),(n,1)]\right)\\
	=&\ f(Q_n) + 2f(Q_n[(1,0),(n,0)]) + f\left(Q_{n-1}+\overline{(1,0)(n-1,0)}+\overline{(1,1)(n-1,1)}\right)\\
	=&\ 2\cdot 7^{n-1} + \frac{1}{14} (8\cdot 7^n - 7\cdot 2^n (3\cdot 2^n + 2)) + f\left(Q_{n-1}+\overline{(1,0)(n-1,0)}+\overline{(1,1)(n-1,1)}\right)
	\end{align*}
	Therefore, $f\left(Q_n+\overline{(1,0)(n,0)}+\overline{(1,1)(n,1)}\right) = 7^n - 2^{2n+1} - 2^{n+1} + 5$. 
	It follows that the answer is the last 10 digits of $7^2021 - 2^{4043} - 2^{2022} + 5$, which is \fbox{7972067500}.
\end{solution}\bigskip
