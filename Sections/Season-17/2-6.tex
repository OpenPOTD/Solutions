\SSbreak\\
\emph{Source: Original}\\
\emph{Proposer: \Ptan}\\
\emph{Problem ID: 244}\\
\emph{Date: 2021-07-31}\\
\emph{Difficulty: Challenging}\\
\SSbreak

\SSpsetQ{
$2021$ distinct rooms are arranged in a circle, each with 4 doorways: two leading to adjacent 
rooms, one leading to the courtyard, and one leading to the outside. However, these doors 
are one-way doors (so you can go from Room A to the courtyard but not from the courtyard to Room A through the same door). 
Find the number of ways to orient the doors such that one can get from any room to any other room. 
}\bigskip

\begin{solution}\hfil\medskip
  
    Solve for general $n$. 
    We convert this into graph format, with $n + 2$ vertices, representing each of $n$ rooms, 
    the courtyard, and the outside; and $3n$ edges, representing the one-way doors; and $2n$ 
    faces. The end result should look like the graph $G$ of a neural network with one input node, 
    one output node, and one hidden layer containing $n$ nodes, with the addition that 
    all $n$ hidden nodes are connected in a (undirected) cycle; another visualization is 
    the gluing of two $n$-gonal pyramids together at their bases. Vertices $u$ and $v$ 
    are connected by a directed edge $u \to v$ iff one can walk from $u$ to $v$ through exactly 
    one door (again, the courtyard and the outside count as rooms). \medskip

    $G$ isn't all too useful. The condition of $G$ being strongly conncted is 
    that $G$ must have a cycle decomposition such that each cycle must be strongly connected, 
    but that's hardly useful, either. Note that $G$ is planar (curve the edge between the topmost 
    and bottommost hidden nodes around eithe one of the start or end nodes, then draw the rest of 
    the edges as common sense dictates); this means it has a dual graph $G'$. We direct the edges of 
    said dual graph as follows: let $u$ be an edge of the original graph, and $u'$ the edge of 
    the dual graph that intersects $u$. Then, rotate $u$ counterclockwise until it coincides with the 
    vertices of $u'$; that is the direction of $u'$. \medskip

    We claim that $G$ is strongly connected iff $G'$ contains no directed cycles. Suppose that 
    $G'$ contained a cycle; this cycle "surrounds" some subset $S$ of vertices in $G$. Now, 
    because the edges in the cycle go either all clockwise or all counterclockwise, 
    this means that all the edges going from $S$ to $G \setminus S$ either are directed 
    into $S$ or out of it. Conversely, if there exists a subset $S$ of vertices in $G$ such that 
    it is impossible to transverse from $S$ to $G \setminus S$, then all the edges from 
    $S$ to $G \setminus S$ either point all in or all out of $S$, meaning that there is a cycle 
    in $G'$. (The motivation here was from the board game \textit{weiqi}; think of black stones 
    as vertices in $S$ and white stones the edges separating $S$ from $G \setminus S$.) \medskip

    Let $A(G)$ be the number of ways to direct the edges of $G$ such that there are no 
    directed cycles in $G$ afterwards. We claim that $A(G) = (-1)^V\chi(G, -1)$, where $V$ 
    is the number of vertices in $G$ and $\chi(G, n)$ is the chromatic polynomial of $G$. 
    ($\chi(G, n)$ counts the number of ways to colour the vertices of $G$ with $n$ colours such 
    that no two adjacent vertices share a colour.)
    More generally, if $A(G, n)$ is the number of ways to: 
    \begin{enumerate}[noitemsep]
        \item label the vertices of $G$ each with an integer between 1 and $n$ inclusive such that if 
        $u \to v$ is a directed edge, then the label assigned to $u$ is no less than the label 
        assigned to $v$ (akin to "ordering" the colours such that some colours are worth more than others)
        \item direct the edges of $G$ such that $G$ is acyclic
    \end{enumerate}
    then $A(G, n) = (-1)^V\chi(G, -n)$. Recall that $\chi(G, n)$ satisfies the following recurrence: 
    $\chi(G, n) = \chi(G\setminus e, n) - \chi(G/e, n),$ where $e$ is an edge, $G\setminus e$ 
    is the graph of $G$ but with $e$ removed, and $G/e$ is the graph of $G$ with the end vertices 
    of $e$ shrunk down into one vertex. Since the values of $A(G, n)$ and $\chi(G, -n)$ are the same 
    at "trivial" graphs (like the one-vertex graph, and combinations of two disconnected graphs) 
    it suffices to prove the recurrence $A(G, n) = A(G\setminus e, n) + A(G/e, n)$. 
    Consider the graph $G\setminus e$. If this graph contains a broken cycle such that adding $e$ 
    would complete the cycle, then adding $e$ in the opposite direction constitutes a valid 
    orienting of $G$; however in exactly $A(G/e, n)$ of these graphs there is no broken 
    cycle that would be completed by adding $e$, so in these graphs both directions of $e$ work. 
\end{solution}\medskip

\begin{solution}[Write-up by \Pqs]\hfil\medskip

    The regions and doorways can be modelled as faces and directed edges of a prism, 
	where one can go from region $A$ to $B$ if the corresponding edge points clockwise w.r. to face $A$. \\ \\
	Suppose there is a directed cycle, then one can only cross from one side of it to the other, 
	thus not fulfilling the criteria of being able to reach any region from any other region. On the contrary, 
	if there are no cycles, then for any nonempty strict subset $S$ of faces, one can always find a face $\notin S$ 
	reachable from some face $\in S$. By selecting a set $\{F\}$ of any face and continuously adding reachable faces to it, 
	one can conclude that any face can be reached from any other face. Thus the answer is the number of ways to 
	direct the edges of a 2021-sided prism to form an acyclic graph. \\ \\
	Let $f(G)$ be the number of ways to direct the edges of an undirected graph $G$ to form an acyclic graph. 
	Let $G\backslash e$ and $G/e$ denote the removal and contraction (removal of edge and merging of 2 vertices) of edge $e$ 
	respectively. Note that $f(G) = 0$ if $G$ contains a loop (edge connecting vertex to itself), 
	and the removal of an edge doesn't change the value of $f$ if there is another edge connecting the same 2 vertices. 
	For some directed graph $G$ with non-loop edge $e$, suppose reversing the direction of $e$ changes whether $G$ is acyclic, 
	then $G\backslash e$ is acyclic and $G/e$ is cyclic. Suppose $G$ is acyclic/cyclic regardless of the orientation of $e$, 
	then both $G/e$ and $G\backslash e$ are acyclic/cyclic. Thus we have $f(G) = f(G/e) + f(G\backslash e)$. \\ \\
	Let $Q_s$ denote an undirected graph in the shape of an $s-1$ by 1 grid of squares,
	containing vertices $(x, y)$ for all pairs of integers $1 \leq x \leq s$, $0 \leq y \leq 1$.
	Let $G+e$ denote the addition of edge $e$ and $G[A,B]$ the merging of vertices $A,B$. Observe $f(Q_1) = 2$ and 
	\begin{align*}
	f(Q_{n+1}) &= f\left(Q_{n+1}\backslash\overline{(n+1,0)(n+1,1)}\right) + f\left(Q_{n+1}/\overline{(n+1,0)(n+1,1)}\right)\\
	&= 4f(Q_n) + (2f(Q_n) + f(Q_n))\\
	&= 7f(Q_n)
	\end{align*}
	Therefore, $f(Q_n) = 2\cdot 7^{n-1}$. Notice $f(Q_2[(1,0),(2,0)]) = 0$, $f(Q_2[(1,0),(2,1)]) = 4$, and
	\begin{align*}
	&\ f(Q_{n+1}[(1,0),(n+1,0)])\\
	=&\ f\left(Q_n+\overline{(1,0)(n,0)}+\overline{(n,1)(n+1,1)}\right) + f\left(Q_n+\overline{(1,0)(n,0)}+\overline{(1,0)(n,1)}\right)\\
	=&\ 3f\left(Q_n+\overline{(1,0)(n,0)}\right) + f(Q_n[(1,0),(n,1)])\\
	=&\ 6\cdot 7^{n-1} + 3f(Q_n[(1,0),(n,0)]) + f(Q_n[(1,0),(n,1)])
	\end{align*}
	\begin{align*}
	&\ f(Q_{n+1}[(1,0),(n+1,1)])\\
	=&\ f\left(Q_n+\overline{(1,0)(n,1)}+\overline{(n,0)(n+1,0)}\right) + f\left(Q_n+\overline{(1,0)(n,0)}+\overline{(1,0)(n,1)}\right)\\
	=&\ 3f\left(Q_n+\overline{(1,0)(n,1)}\right) + f(Q_n[(1,0),(n,0)])\\
	=&\ 6\cdot 7^{n-1} + 3f(Q_n[(1,0),(n,1)]) + f(Q_n[(1,0),(n,0)])
	\end{align*}
	Thus, $f(Q_n[(1,0),(n,0)]) = \frac{1}{28} (8\cdot 7^n - 7\cdot 2^n (3\cdot 2^n + 2))$.
	Finally, $f\left(Q_2+\overline{(1,0)(2,0)}+\overline{(1,1)(2,1)}\right) = 14$ and
	\begin{align*}
	&\ f\left(Q_n+\overline{(1,0)(n,0)}+\overline{(1,1)(n,1)}\right)\\
	=&\ f\left(Q_n+\overline{(1,0)(n,0)}\right) + f\left(Q_n+\overline{(1,0)(n,0)}[(1,1),(n,1)]\right)\\
	=&\ f(Q_n) + 2f(Q_n[(1,0),(n,0)]) + f\left(Q_{n-1}+\overline{(1,0)(n-1,0)}+\overline{(1,1)(n-1,1)}\right)\\
	=&\ 2\cdot 7^{n-1} + \frac{1}{14} (8\cdot 7^n - 7\cdot 2^n (3\cdot 2^n + 2)) + f\left(Q_{n-1}+\overline{(1,0)(n-1,0)}+\overline{(1,1)(n-1,1)}\right)
	\end{align*}
	Therefore, $f\left(Q_n+\overline{(1,0)(n,0)}+\overline{(1,1)(n,1)}\right) = 7^n - 2^{2n+1} - 2^{n+1} + 5$. 
	It follows that the answer is the last 10 digits of $7^{2021} - 2^{4043} - 2^{2022} + 5$, which is \fbox{7972067500}.
\end{solution}\bigskip
