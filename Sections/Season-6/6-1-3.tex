  
\SSbreak\\
\emph{Source: \Cop}\\
\emph{Proposer: \Paiya}\\ %\Pchan \Pbrain \Pss
\emph{Problem ID: 90}\\
\emph{Date: 2021-01-06}\\
\SSbreak

\SSpsetQ{
Let \(f:[0,1]\to\mathbb{R}\) be a function with the following properties:
\begin{itemize}
	\item \(f(x)+f\left(\frac{1}{2}\right)f(1-x)=2f\left(\frac{1}{2}\right)\)\\
	\item \(2f(x)=f(3x)\)\\
	\item \(f\) is non-decreasing
\end{itemize}

Then the sum of the possible values of \(f\left(\frac{4}{5}\right)+1\) can be written in the form \(\frac{m}{n}\), where \(m\) and \(n\) are relatively prime positive integers. What is the value of \(100m+n\)?
	%Put Problem Here
}\bigskip

\begin{solution}[Solution by \Paiya]\hfil\medskip
	
	First, find some basic values: $2f(0) = f(0) \iff f(0) = 0$ and $f(1/2) + f(1/2)^2 = 2f(1/2) \iff f(1/2) = 0, 1$. Now if $f(1/2) = 0$ note that $f(1) = 0$; since $f$ is nondecreasing $f$ is the zero function. Now for $f(1/2) = 1$: $f(x) + f(1 - x) = 2 \iff f(1) = 2$. It's fruitless to get $4/5$ from 0, 1, and 1/2 by repeated operations of $1 - x$ and $x/3$; however note that $f(1/3) = 1 \iff f(2/3) = 1$ and since $f$ is nondecreasing all values of $x$ between 1/3 and 2/3 inclusive will have $f(x) = 1$. Then finding $f(3/5)$ is easy: it's just 1, so $f(1/5) = 1/2 \iff f(4/5) = 3/2$. Our possible values of $g(4/5)$ are $1$ and $5/2$. This gives us an answer of \(\fbox{702}\)
\end{solution}\bigskip