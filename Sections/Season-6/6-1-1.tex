\SSbreak\\
\emph{Source: \Csmc, 2015 Q18}\\
\emph{Proposer: \Pss}\\
\emph{Problem ID: }\\
\emph{Date: }\\
\SSbreak

\SSpsetQ{

What is the largest integer \(k\) for which \(k^2\) is a factor of \(10!\)?

}\bigskip

\begin{solution}\hfil\medskip
	
Let us write the factorisation of 10! in a table, so we can see  the indexes of the powers.

	\begin{table}[h!]
		\centering
		\begin{tabular}{c|cccc}
		   & 7 & 5 & 3 & 2 \\ \hline
		2  & 0 & 0 & 0 & 1 \\
		3  & 0 & 0 & 1 & 0 \\
		4  & 0 & 0 & 0 & 2 \\
		5  & 0 & 1 & 0 & 0 \\
		6  & 0 & 0 & 1 & 1 \\
		7  & 1 & 0 & 0 & 0 \\
		8  & 0 & 0 & 0 & 3 \\
		9  & 0 & 0 & 2 & 0 \\
		10 & 0 & 1 & 0 & 1
		\end{tabular}
		\end{table}

We want to find \(k^2\), where \(k\) is maximised. That means we want to add as many of the indexes together such that they all end up even. From inspecting the table, we see that they all add up to an even number so long as we don't include the 7. Thus \(k^2=\frac{10!}{7}=5^2\cdot3^4\cdot2^8\), this gives an answer of \(k=\fbox{720}\)
\end{solution}\bigskip
