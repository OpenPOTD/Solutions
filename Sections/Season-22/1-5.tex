\SSbreak\\
\emph{Source: Unknown}\\
\emph{Proposer: \Pchris}\\
\emph{Problem ID: 294}\\
\emph{Date: 2022-01-07}\\
\emph{Difficulty: Hard}\\
\SSbreak

\SSpsetQ{

You're trying to open a peculiar lock. The lock consists of a circular ring of $n$ equally-spaced 
indistinguishable buttons, which you can see, that covers a ring of $n$ pins, which you cannot see. 
Each button covers a pin, and pushing a button toggles the pin underneath it between 
\textit{engaged} and \textit{disengaged}. The lock opens when all $n$ pins are disengaged. \medskip 

You can attempt to open the lock by simultaneously pressing any number of buttons. If all pins 
are disengaged as a result, the lock opens. Otherwise, the ring of pins rotates to an arbitrary 
position; the buttons might cover different pins, but the state of each pin is retained. \medskip

As complicated as this lock may seem, for certain values of $n$ it is always possible to open 
such a lock, no matter the initial state of the lock. Find all such values of $n$, and find 
the most number of attempts you have to make to guarantee that the lock will open. 
}\bigskip

\begin{solution}\hfil\medskip
  
    $n$ must be a \fbox{power of 2}. The worst-case scenario requires $\boxed{2^n - 1}$ attempts.\medskip

    Experiment with small values of $n$ first. $n = 1$ is trivial: just press the button. 
    For $n = 2$, the lock could start with either all pins engaged, or only one. To rule out 
    the all pins engaged case, press both buttons; if you fail then that means there is 
    exactly one pin engaged. In this case, press one button; if you still fail then that means 
    you engaged a disengaged pin (otherwise you would've unlocked it). Thus, pressing both 
    buttons guarantees an unlock. As you can see, the worst-case scenario required $2^2 - 1 = 3$ 
    attempts, which goes through all possible locked pin states.\medskip

    After trying with $n = 3$ for a while, you'll probably fail to find a way to solve the 
    cases where either one or two pins are engaged. Indeed, $n = 3$ is impossible. Note that 
    since $n$ is odd, for every attempt there exists a pair of adjacent pins that are either 
    both toggled or both not toggled. (If not, it means you were able to push every other 
    button, implying that $n$ is even.) If, for every attempt you make, this pair has one 
    pin engaged and one pin disengaged, you will never be able to open the lock, since there's 
    no way to match the pin states of that pair. The following is such a scenario: such a 
    mismatched pair exists, and due to bad luck you always engage both pins of that pair during 
    each attempt you make. This argument extends to all $n$ such that $n$ has an odd factor $d$: 
    choose $d$ equally spaced pins, and define adjacency by ignoring all unchosen pins. \medskip

    The same idea of pin-pairing gets us an algorithm for unlocking $n = 2^k$. We've already 
    seen the algorithm for $n = 2^1$, and can extend it to $n = 2^2$ as follows. Imagine separating 
    the pins into two pairs, then matching the state of pins in each pair. Then, the lock 
    reduces down to the $n = 2$ case. However, we have no way of knowing whether each pair 
    is matched or not, so we have to go through all $2^{n/2} = 4$ states of pair-matchedness, 
    trying the $n = 2$ algorithm each time. To match or unmatch a pair, push exactly one button 
    of the pair; thus, going through all pair-matchedness states is the same as unlocking the $n = 2$ 
    lock, since for the $n = 4$ case there are two pins we need to toggle to go through all pair-matchedness 
    states and the $n = 2$ unlocking algorithm goes through all pin states. But we can't just 
    pair up adjacent buttons, as those would get confused under rotation. Instead, pair up 
    diametrically opposite buttons, as those stay invariant under rotation. The algorithm 
    is as follows: 

    \newpage\begin{itemize}[noitemsep]
        \item Pair up diametrically opposite buttons. 
        \item Apply the $n = 2$ unlocking algorithm. 
        \item Apply the first move in the $n = 2$ unlocking algorithm. 
        \item Apply the $n = 2$ unlocking algorithm.
        \item Apply the second move in the $n = 2$ unlocking algorithm.
        \item Apply the $n = 2$ unlocking algorithm.
        \item Apply the third move in the $n = 2$ unlocking algorithm.
        \item Apply the $n = 2$ unlocking algorithm.
    \end{itemize}

    This algorithm can be extended; for $n = 2^k$ replace $2$ with $2^{k - 1}$ and the same applies. 
    Note that this algorithm has a worst-case of $2^n - 1$. Indeed, this is the best we can do: since we'll 
    never be able to tell the exact state of any pin, the best we can do is systematically iterate 
    through every possible pin state, of which there are $2^n - 1$. 
\end{solution}\bigskip
