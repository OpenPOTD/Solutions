\SSbreak\\
\emph{Source: IMTS Year 1991 R4 Problem 5 (modified)} \\
\emph{Proposer: \Pflame}\\ %\Pchan \Pbrain \Pss
\emph{Problem ID:}\\
\emph{Date: }\\
\emph{Difficulty: Easy}\\
\SSbreak

\SSpsetQ{
	Paper $\triangle XYZ$ has $XY=69$, $YZ=91$, $ZX=100$. Let $A$, $B$, $C$ be the midpoints of $\overline{XY}$, $\overline{YZ}$, $\overline{ZX}$ respectively. The triangle is folded about $AB$, $BC$, and $CA$ to form a tetrahedron $\mathcal{T}$. What is the volume of $\mathcal{T}$?
	%Put Problem Here
}\bigskip

\begin{solution}\hfil\medskip
	
	\href{https://artofproblemsolving.com/community/q2h54423p339658}{AoPS Solution using Pythagorean Theorem} \medskip

	We will prove that in an isosceles tetrahedron (opposite edges have the same length) with edge lengths $a, b, c$, the volume of such a tetrahedron is 
	$$V = \sqrt{\dfrac{\left(a^2 + b^2 - c^2\right)\left(a^2 - b^2 + c^2\right)\left(-a^2 + b^2 + c^2\right)}{72}}.$$
	First, note that every "acute" tetrahedron can be inscribed in a parallepiped prism, with each edge being a diagonal of each face (see \href{https://www.cut-the-knot.org/triangle/TetrahedronInParallelepiped.shtml}{here} for a picture). 
	This is because there is exactly one pair of parallel planes that pass through each pair of opposite edges 
	(since opposite edges are skew lines, so you can't just rotate the planes wrt the axis of one edge). 
	Also note that since opposite edges are skew, this means that the opposite edges take up the two different diagonals of their inscribed parallelogram face.
	Since opposite edges are equal in an isosceles tetrahedron, this implies that the parallelogram has equal diagonals, implying that it's a rectangle,
	and the circumscribed parallepiped is actually a rectangular prism. Letting $x, y, z$ represent the edges of the prism, we have
	$$x^2 + y^2 = a^2, y^2 + z^2 = b^2, z^2 + x^2 = c^2 \iff \left(x^2, y^2, z^2\right) = \left(\dfrac{-a^2 + b^2 + c^2}{2}, \dfrac{a^2 - b^2 + c^2}{2}, \dfrac{a^2 + b^2 - c^2}{2}\right)$$
	and now we claim that the volume of the prism is three times the volume of the tetrahedron; indeed the space outside the inscribed tetrahedron but inside the prism
	is composed of three congruent right-angle-cornered tetrahedra, each with volume $\frac{xyz}{6}$. 
	Thus $$V = \dfrac{xyz}{3} = \sqrt{\dfrac{\left(a^2 + b^2 - c^2\right)\left(a^2 - b^2 + c^2\right)\left(-a^2 + b^2 + c^2\right)}{72}}.$$
	
	Answer: \(\boxed{7605}\)
	%Put sol here
\end{solution}\bigskip
