\SSbreak\\
\emph{Source: \Csmc, 2015 Q4}\\
\emph{Proposer: \Pbrain}\\
\emph{Problem ID: 21}\\
\emph{Date: 2020-10-27}\\
\SSbreak

\SSpsetQ{
    Consider the positive integer \(N\), and Two internally tangent circles \( \Gamma_1 \) and \( \Gamma_2 \) 
    are given such that \( \Gamma_1 \) passes through the center of \( \Gamma_2 \). 
    Find the fraction of the area of \( \Gamma_1 \) lying outside \( \Gamma_2 \). 
    If this fraction is \( \frac ab \) where \( \gcd(a, b) = 1\), then find \( a+b \).
}\bigskip

\begin{solution}\hfil\medskip

    Suppose $\Gamma_2$ has radius $2r$. Since $\Gamma_1$ is internally tangent to $\Gamma_2$ and passes through its centre, the radius of $\Gamma_1$ is half the radius of $\Gamma_2$, 
    i.e. just $r$. So fraction of the area of $\Gamma_1$ lying outside $\Gamma_2$ is $\dfrac{4\pi r^2 - \pi r^2}{4 \pi r^2} = \dfrac{3}{4}$. 
    Since $3 + 4 = 7$, the answer is $\boxed{7}$
\end{solution}\bigskip