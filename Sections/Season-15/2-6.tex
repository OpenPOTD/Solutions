\SSbreak\\
\emph{Source: 2020 Iran MO R3G4}\\
\emph{Proposer: \Pchris}\\ %\Pchan \Pbrain \Pss
\emph{Problem ID: 216}\\
\emph{Date: 2021-06-19}\\
\emph{Difficulty: Challenging}\\
\SSbreak
 
\SSpsetQ{
    $\triangle PQR$ has $PQ = 6, QR = 7, RP = 8$, incentre $I$, and circumcentre $O$. The external angle bisector of $P$ meets $\overleftrightarrow{QR}$ at $S$, 
    and $I_Q$ is the $Q$-excentre. The point $T$ is chosen along $\overleftrightarrow{IQ}$ such that $QT = 2IQ$ and $IT > IQ.$ Let $F$ be the point on the circumcircle
    of $\triangle DI_QT$ such that $DF$ is a diameter. Then the perimeter of $\triangle OIF$ can be expressed in the form $\frac{a + b \sqrt{c}}{d}$,
    where $\gcd(a, b, d) = 1$, $a, b$ are integers, and $c, d$ are positive integers. Find $1000a + 100b + 10c + d.$
    %Put Problem Here
}\bigskip

\begin{solution}\hfil\medskip

    \href{https://artofproblemsolving.com/community/c6h2347106p18981757}{Solution} so $O, I, F$ are collinear; using $OI^2 = R^2 - 2Rr$ we get $IF = \frac{16 \sqrt{15}}{15} \iff \boxed{3530}.$
    %Put sol here
\end{solution}\bigskip
